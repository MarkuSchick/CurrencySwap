\documentclass[11pt]{beamer}
% \documentclass[11pt,handout]{beamer}
\usepackage[T1]{fontenc}
\usepackage[utf8]{inputenc}
\usepackage{float, afterpage, rotating, graphicx}
\usepackage{epstopdf}
\usepackage{longtable, booktabs, tabularx}
\usepackage{fancyvrb, moreverb, relsize}
\usepackage{eurosym, calc}
\usepackage{amsmath, amssymb, amsfonts, amsthm, bm}
\usepackage{hyperref}
\usepackage[
    natbib=true,
    bibencoding=inputenc,
    bibstyle=authoryear-ibid,
    citestyle=authoryear-comp,
    maxcitenames=3,
    maxbibnames=10,
    useprefix=false,
    sortcites=true,
    backend=biber
]{biblatex}
\AtBeginDocument{\toggletrue{blx@useprefix}}
\AtBeginBibliography{\togglefalse{blx@useprefix}}
\setlength{\bibitemsep}{1.5ex}
\addbibresource{refs.bib}



\hypersetup{colorlinks=true, linkcolor=black, anchorcolor=black, citecolor=black, filecolor=black, menucolor=black, runcolor=black, urlcolor=black}

\setbeamertemplate{footline}[frame number]
\setbeamertemplate{navigation symbols}{}
\setbeamertemplate{frametitle}{\centering\vspace{1ex}\insertframetitle\par}


\begin{document}

\title{Currency Swap}

\author[Markus Schick]
{
{\bf Markus Schick}\\
{\small University of Bonn}\\[1ex]
}


\begin{frame}
    \titlepage
    \note{~}
\end{frame}

\begin{frame}[t]
    \frametitle{Background}
    \begin{itemize}
        \item This is the final project of the course \textit{Effective Programming}
        \item Work has be done using the template of \citet{GaudeckerEconProjectTemplates}
    \end{itemize}
    \note{~}
\end{frame}



\begin{frame}[t]
    \frametitle{EURO/USD exchange rate}
    \begin{itemize}
        \item EUR/USD exchange rate fluctuated between 0.8 and 1.6
        \item daily returns are very low with phases of volatility (e.g. in 2008)
    \end{itemize}
    \begin{figure}
        \caption{EUR/USD exchange rate volatility}
        \includegraphics[width=\textwidth]{../../bld/figures/euro_usd_timeseries}
    \end{figure}
    \note{~}
\end{frame}

\begin{frame}[t]
    \frametitle{But how large is the change over 1 year?}
    \begin{itemize}
        \item We take a completely non-parameteric approach
        \item 1) We look of the sample of 1 year periods from 1990-Today (Historical)
        \item 2) We (bootstrapp)[] 1 year samples from historical data (Bootstrapping)
        \item We use the \bf{stationary bootstrap}
    \end{itemize}
    \note{~}
\end{frame}

\begin{frame}[t]
    \frametitle{Stationary Bootstrap}
    \begin{itemize}
        \item keeps the autocorrelated structure of our data intact
        \item autocorrelation in variance plays a prominent role in financial data (GARCH modelling)
        \item the recombinator package does the job for us
        \item optimal (average) interval length is calculated internally
    \end{itemize}
    \note{~}
\end{frame}

\begin{frame}[t]
    \frametitle{Simulated 1-year EURO/USD exchange rate returns (Bootstrapping)}
    \begin{figure}
        \caption{Sample created with Bootstrapping}
        \includegraphics[width=80mm]{../../bld/figures/euro_usd_bootstrapped}
    \end{figure}
    \note{aggregated returns from 262 trading days}
\end{frame}

\begin{frame}[t]
    \frametitle{Simulated 1-year EURO/USD exchange rate returns (Historical)}
    \begin{figure}
        \caption{Sample created with historical exchange rate}
        \includegraphics[width=80mm]{../../bld/figures/euro_usd_historical}
    \end{figure}
    \note{aggregated returns from 262 trading days}
\end{frame}

\begin{frame}[t]
    \frametitle{What is DEFI?}
    \begin{itemize}
        \item offers financial services without relying on (trusted) financial intermediaries
        \item uses smart-contracts running on a blockchain (majority runs on \href{https://ethereum.org/en/}{\bf{Ethereum}})
        \item completely transparent and efficient, but risky (contracts can be hacked)
        \item \href{https://en.wikipedia.org/wiki/Decentralized_finance}{Wikipedia article}, \href{https://research.stlouisfed.org/publications/review/2021/02/05/decentralized-finance-on-blockchain-and-smart-contract-based-financial-markets}{Economic paper}
        \item Feel free to reach out if you wanna know more ;)
    \end{itemize}
    \note{~}
\end{frame}

\begin{frame}[t]
    \frametitle{The Problem}
    \begin{itemize}
        \item Algorithmic savings protocols such as \href{https://app.aave.com/}{\bf{Aave}} offer good returns
        \item require users to hold stable coins (denominated in USD)
        \item this results in exchange rate risks for users from outside of the US
        \item worked on fixing this problem on some \href{https://ethglobal.online/}{Ethereum Hackathons}
        \item Now I want to do some simulations of how the current economic design would perform
        \item and create a framework to quickly iterate on the version
    \end{itemize}
    \note{~}
\end{frame}


\begin{frame}[t]
    \frametitle{Introduce a swap contract}
    \begin{itemize}
        \item Alice goes long Euro by putting 1 Dollar into a smart contract and receiving a token called EURlong
        \item Bob goes short Euro by putting 1 Dollar into a smart contract and receiving a token called EURshort
        \item After 1 year the contract allocates the collateral according to below formula
    \end{itemize}
    \begin{align*}
    EURlong =  1 + \frac{e_1 - e_0}{e_0} * leverage  \\
    EURshort =  collateral - EURlong
\end{align*}


    \note{~}
\end{frame}

\begin{frame}[t]
    \frametitle{Example calculations}
    \begin{itemize}
        \item for $e_0 = 1, e_1 = 1.05, leverage = 2, collateral = 2$
        \item Alice receives 1.5 USD back
        \item Bob receives 0.5 USD back
        \item Why might Alice and Bob take that deal?
        \item Alice can transform USD assets into EUR assets with EURlong
        \item Bob can transform USD debt into EUR debt with EURshort
    \end{itemize}

    \note{~}
\end{frame}

\begin{frame}[t]
    \frametitle{Simulation analysis}
    \begin{itemize}
        \item We know simulate the payoff of the EURlong and EURshort certificate with simulated 1 year returns from before
        \item In the following I will just show bootstrapped results. Both graphs are produced and are very similar.
        \item We will show density plots of payouts in all generated scenarios
    \end{itemize}

    \note{~}
\end{frame}

\begin{frame}[t]
    \frametitle{Number of runs with negative payout}
    \begin{itemize}
        \item depending on the exchange rate movement, leverage and asset allocation we might end up with negative income
        \item this is problematic since we can \it{typically} not enforce demands on a blockchain
        \item We hereby try to minimize the probability of this happening
    \end{itemize}

    \note{~}
\end{frame}


\begin{frame}[t]
    \frametitle{Swap contract payout depending on asset allocation and leverage}
    \begin{figure}
        \caption{Density of payouts}
        \includegraphics[width=70mm]{../../bld/figures/bootstrapped_negative_payout}
    \end{figure}

\end{frame}


\begin{frame}[t]
    \frametitle{Payout profile of EURO short swap contract}
    \begin{figure}
        \caption{Sample created with Bootstrapping}
        \includegraphics[width=80mm]{../../bld/figures/bootstrapped_eurshort_payout}
    \end{figure}
\end{frame}


\begin{frame}[t]
    \frametitle{Payout profile of EURO long swap contract}
    \begin{figure}
        \caption{Sample created with Bootstrapping}
        \includegraphics[width=80mm]{../../bld/figures/bootstrapped_eurlong_payout}
    \end{figure}

\end{frame}

\begin{frame}[t]
    \frametitle{Expected EUR return on collateral locked}
    \begin{figure}
        \caption{Sample created with Bootstrapping}
        \includegraphics[width=80mm]{../../bld/figures/bootstrapped_expected_payout_EUR}
    \end{figure}
\end{frame}

\begin{frame}[t]
    \frametitle{Expected USD return on collateral locked}
    \begin{figure}
        \caption{Sample created with Bootstrapping}
        \includegraphics[width=80mm]{../../bld/figures/bootstrapped_expected_payout_USD}
    \end{figure}
\end{frame}

\begin{frame}[t]
    \frametitle{Conclusion}
    \begin{itemize}
        \item leverage factor above 5 must be considered to be unsafe
        \item higher proportion of assets invested in USD increase payout of EURshort
        \item higher proportion of assets invested in USD increases chance of negative payout
    \end{itemize}

    \note{~}
\end{frame}

% Print black screen only in presentation mode for finishing up.
\mode<beamer> {
    \beamersetaveragebackground{black}
    \begin{frame}
        \frametitle{}
    \end{frame}

    \beamersetaveragebackground{white}
}

\begin{frame}[allowframebreaks]
    \frametitle{References}

    \renewcommand{\bibfont}{\normalfont\footnotesize}
    \printbibliography

\end{frame}

\end{document}
